%%%%%%%%%%%%%%%%%%%%%%%%%%%%%%%%%%%%%%%%%
% Medium Length Professional CV
% LaTeX Template
% Version 2.0 (8/5/13)
%
% This template has been downloaded from:
% http://www.LaTeXTemplates.com
%
% Original author:
% Trey Hunner (http://www.treyhunner.com/)
%
% Important note:
% This template requires the resume.cls file to be in the same directory as the
% .tex file. The resume.cls file provides the resume style used for structuring the
% document.
%
%%%%%%%%%%%%%%%%%%%%%%%%%%%%%%%%%%%%%%%%%

%----------------------------------------------------------------------------------------
%   PACKAGES AND OTHER DOCUMENT CONFIGURATIONS
%----------------------------------------------------------------------------------------

\documentclass{resume} % Use the custom resume.cls style

\usepackage[left=0.4 in,top=0.3 in,right=0.4 in,bottom=0.3in]{geometry} % Document margins
\newcommand{\tab}[1]{\hspace{.2667\textwidth}\rlap{#1}}
\newcommand{\itab}[1]{\hspace{0em}\rlap{#1}}
\name{Andrew Lyjak} % Your name
\address{6205 N Oakland Ave, Indianapolis IN, 46220} % Your address
\address{+1 734 604 6163 \\ alyjak@gmail.com} % Your phone number and email

\begin{document}

%----------------------------------------------------------------------------------------
%   EDUCATION SECTION
%----------------------------------------------------------------------------------------

\begin{rSection}{Education}

{\bf Bachelor of Engineering in Aerospace Engineering} \hfill {2005 - 2009} \\
Minor in German \smallskip \\
\\
University of Michigan, GPA: 3.456/4.000

{\bf Masters of Space Systems Engineering} \hfill {2009 - 2010}
\\
University of Michigan, GPA: 7.538/8.000

\end{rSection}

%----------------------------------------------------------------------------------------
%   TECHNICAL STRENGTHS SECTION
%----------------------------------------------------------------------------------------

\begin{rSection}{Interests}

  Design and Development Processes, Distributed Systems, Resilient Systems, Autonomy, Automation,\\
  Software Certification, Verification of Complex Systems, Model Driven Development, Continuous Integration,\\
  Agile Development

\end{rSection}

\begin{rSection}{Skills and Tools}
Software Development Processes, Software Process Audits, System Certification,
Verification Planning, Fault Tolerance Analysis, Python, Javascript, HTML, CSS,
Bash, Linux, R, C, C++, SQL, \LaTeX, ReStructuredText, Graph Analysis, Graph
Databases, git, Trac, Subversion, JIRA, Confluence
\begin{tabular}{ @{} >{\bfseries}l @{\hspace{6ex}} l }
\end{tabular}

\end{rSection}

%-------------------------------------------------------------------------------
%   PROJECTS

\begin{rSection}{SpaceX Software Reliability Engineering}

\begin{rSubsection}{Software Certification for NASA Safety and Quality Requirements} {2010 - Present}{}{}

Certification activities applicable to Cargo Dragon for the NASA COTS,
and CRS contracts

\item Wrote the Flight Software Development Plan. This document defines SpaceX's
  custom agile development process for designing, developing, and verifying
  SpaceX flight software. (2011)
\item Designed and and wrote the Cargo Dragon Flight Software ``Computer Based
  Control System'' certification documentation. This material was used to
  demonstrate SpaceX compliance with NASA software safety requirements and also
  defined many of the software safety test and analysis activities executed for
  verification of software safety. (2011)
\item Performed Flight Software Process Audits to verify software quality, and
  compliance with NASA requirements. (2012)
\item Maintained the above certifications and documents through all changes to
  system design and software (2012 through present).

\end{rSubsection}

\begin{rSubsection}{Independent Verification and Validation (IVV) Contract Management (2010 - Present)}{}{}

\item Serve as the technical point of contact for SpaceX's IVV Contracts for
  safety critical software. In this role, ensure that the contractor is provided
  sufficient data to effectively evaluate the safety of the flight software with
  respect to its design, source code, and verification results, and ensure that
  their feedback is addressed and incorporated to create a better product. The
  contracts include IVV coverage of Crew Dragon, Cargo Dragon, and SpaceX's
  Autonomous Flight Termination System (AFTS).

\end{rSubsection}

%------------------------------------------------

\begin{rSubsection}{Software Process Tool Development}{2013 - 2014}{}{}

\item branchdiff - an application to view the differences between to Subversion
  branches to facilitate merge decisions between them. Displays differences as
  commits or as the set tickets referenced within those commit messages.
\item Developed an application for viewing change over time for various Trac
  ticket queries.
\item Performed trade studies on various software development ticketing systems
  -- Trac, JIRA, Phabricator, Redmine.

\end{rSubsection}


%-------------------------------------------------


\begin{rSubsection}{Design and Development of the SpaceX Software Standard}{2013-2015}{}{}

\item Developed an internal software engineering standard. The standard provides
  a set of requirements that can be used for different classifications of
  software development, classifications include A, for safety and mission
  critical software, to D, used for desktop, R\&D, or other non-critical
  applications. The standard requirement are divided into categories for
  different aspects of software design and development, ranging from change
  management, risk management, verification processes, and technical and
  analytical requirements for safety critical software. The standard is
  currently on version 1.3 and is used for all softare related to the NASA
  Commercial Crew contract.

\end{rSubsection}

%--------------------------------------------------

\begin{rSubsection}{Design, Development, and Execution of the SpaceX Fault\\
    Tolerance Analysis Process}{2016-present}{}

\item In tandem with another engineer at SpaceX, developed the Fault Tolerance
  Analyis Process. This process is used to evaluate autonomous and piloted
  electromechanical control systems to systematically evaluate their fault
  tolerance and develop adequate fault detection, isolation, and recovery
  algorithms and procedures such that each systems' health and redundant
  capabilities are successfully managed.
\item Developed test and analysis plans for testing fault tolerance as defined
  through the Fault Tolerance Analysis products.
\item With a team of 6 other SpaceX Engineers, executed the Fault Tolerance
  Analysis Process against 25 separate autonomous control systems present
  within the Crew Dragon Architecture over the course of a year.

\end{rSubsection}

%--------------------------------------------------

\begin{rSubsection}{Incorporating Software Development Processes within Systems Design Processes}{2015-present}{}{}

\item Installed, Modified, and Adminstered an internal fork of ReadTheDocs for
  use within the SpaceX Intranet (2015-Present). Now used for design
  documentation by over 160 internal projects
\item Tracegraph - Design and Development a protocol for defining systems
  relationships. Protocol includes data format, markup, execution algorithm,
  visualization layer. The protocol was developed to simplify requirements and
  verification traceability, and system dependency management. The protocol can
  be used for impact analysis for existing designs or prototyping and simulation
  of initial complex system designs.
\end{rSubsection}

\end{rSection}


%---------------------------------------------------------------------------------
%  DECLARATION
%--------------------------------------------------------------------------------

\begin{rSection}{ Declaration  } \itemsep -3pt

\item I hereby declare that all the details furnished above are true to the best of my knowledge and belief.

\end{rSection}
\end{document}
