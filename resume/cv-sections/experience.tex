% ----------------------------------------------------------------------------------------
%	SECTION TITLE
% ----------------------------------------------------------------------------------------

\cvsection{Experience}

% ----------------------------------------------------------------------------------------
%	SECTION CONTENT
% ----------------------------------------------------------------------------------------


\begin{cventries}

  % ------------------------------------------------

  \cventry
  {A Lyjak Cybernetics Consulting LLC}{Owner}{Ann Arbor, MI}{2023 - Present}
  {
    \begin{cvitems}
    \item Consultation services related to Production Lifecycle Management tools and processes for
      businesses looking to coherently integrate multiple domain-specific workflows
      (i.e. production, sales, and engineering) into their enterprise application suite.
    \item Feature development, maintenance, testing for H3D production workflow service.
    \item Proposal development for Venturi Astrolab's NASA Lunar Vehicle RFP response. Helped design
      Astrolab's certification and verification strategy for a human-ready lunar vehicle.
    \end{cvitems}
  }

  \cventry
  {H3D, Inc.}{Software Release Quality Engineer}{Ann Arbor, MI}{2020 - 2023}
  {
    \begin{cvitems}
    \item Designed, implemented, tested, and released a production workflow service using
      Docker, Ansible, Django, Webpack. The application uses configuration as code principles in
      order to re-deploy with a single command.
    \item Designed, and implemented a client-side distributed network procedure application intended
      to seamlessly integrate human decision making with configurable automation
      capabilities. Programmed in Rust.
    \end{cvitems}
  }

  \cventry
  {Raytheon}{Senior Systems Engineer II}{Indianapolis, IN}{2019 - 2020}
  {
    \begin{cvitems}
    \item Designed, implemented, tested, and released a continuous integration
      platform. Architecture relies on on-premise cloud, managed through Ansible, and relies on
      Jenkins for performing unit-test, documentation, and integration testing of a
      multi-repository, multi-platform, multi-language hardware+software product.
    \end{cvitems}
  }

  \cventry
  {SpaceX}{Software Mission Assurance Engineer}{Hawthorne, CA}{2010 - 2018}
  {}

  \cventry
  {Software Certification for NASA Safety and Quality Requirements}
  {}{}{2010 - 2018}
  {
    \begin{cvitems}
    \item Coordinated with SpaceX and NASA engineering and management teams to define and approve
      the SpaceX Flight Software Development Process. Was the primary author, maintainer, and
      auditor for the \textbf{Flight Software Development Plan}, which describes the process. This
      document defines SpaceX's agile development process for designing, developing, and verifying
      flight software.
    \item Coordinated with SpaceX and NASA engineering teams to document the core flight software
      architecture in order to demonstrate its fault tolerance and intrinsic safety. Was the primary
      author and maintainer of the Cargo Dragon Flight Software \textbf{Computer Based Control
      Systems} documentation. This material is used to demonstrate compliance with NASA software
      safety requirements and also defines many of the software safety test and analysis activities
      executed for verification of software safety related to the Dragon cargo vehicle.
    \end{cvitems}
  }

  % ------------------------------------------------

  \cventry
  {Independent Verification and Validation (IVV) Contract Management}{}{}{2010 - 2018}
  {
    \begin{cvitems}
    \item Served as the technical point of contact for SpaceX's IVV Contracts for safety critical
      software, in this role I was responsible for communicating and coordinating exchanges between
      SpaceX Engineers and the third party assessor for every mission to the International Space
      Station.
    \item Managed separate IVV contracts associated with independent assessment of the safety of
      flight software for Crew Dragon, Cargo Dragon, and the Autonomous Flight Termination System.
    \end{cvitems}
  }

  % ------------------------------------------------

  \cventry
  {Software Process Tool Development}{}{}{2012 - 2018}
  {
    \begin{cvitems}
    \item \textbf{branchdiff} - Developed an application to view the differences between two
      Subversion branches to facilitate merge decisions between them. Displays differences as
      commits or as the set tickets referenced within those commit messages.
    \item Developed an application for viewing change over time for various Trac ticket queries.
    \item Performed trade studies on various software development ticketing systems. The study
      factored into SpaceX's decision to adopt JIRA across multiple business domains.
    \item \textbf{ReadTheManual} - Installed, modified, and administered an internal fork of
      ReadTheDocs for use within the SpaceX intranet. This service is used to build and distribute
      documentation for over 160 internal projects.
    \item \textbf{Tracegraph} - Designed and developed a library for defining systems relationships
      across information housed within multiple data silos. The library is used for verification
      tracking purposes to support compliance tracking of SpaceX processes to customer requirements.
    \end{cvitems}
  }

  % ------------------------------------------------

  \cventry
  {Design and Development of the SpaceX Software Standard}{}{}{2013 - 2018}
  {
    \begin{cvitems}
    \item Coordinated with SpaceX and NASA engineering and management teams to define and approve
      the \textbf{SpaceX Software Standard} as a controlling standard for the Commercial Crew
      Contract. The standard provides a set of requirements to be applied to different
      classifications of software development, classifications include A, for safety and mission
      critical software, through D, used for desktop, R\&D, or other non-critical applications. The
      standard is used to evaluate the quality of all software processes related to the Commercial
      Crew system.
    \end{cvitems}
  }

  % ------------------------------------------------

  \cventry
  {Fault Tolerance Analysis}{}{}{2016 - 2018}
  {
    \begin{cvitems}
    \item Co-developed the \textbf{Fault Tolerance Analysis Process}. This process is used to
      evaluate autonomous and/or operator controlled electromechanical systems. Analysis is used to
      assess the fault tolerance of a design in order to determine its capabilities and develop
      fault detection, isolation, and recovery logic for managing redundant capabilities.
    \item Developed test and analysis plans for verifying fault tolerance as defined through the
      Fault Tolerance Analysis products.
    \item With a team of 6 other engineers, executed the Fault Tolerance Analysis Process against 25
      separate autonomous control systems within the Crew Dragon Architecture over the course of a
      year.
    \end{cvitems}
  }
  % ------------------------------------------------
\end{cventries}
